\documentclass{article}

\usepackage{polski}
\usepackage[utf8]{inputenc}
\usepackage{graphicx}
\usepackage{float}

\author{Tomasz Biegus}
\title{Sprawozdanie 1 z przedmiotu Statystyczna Analiza Danych}
\begin{document}
\maketitle



\section{Analiza}
\subsection{Sprawdzenie danych}
Przed przystąpieniem do analizy należy sprawdzić czy w danych nie występują wartości, które z pewnością są wynikiem błędu. Zwykle pewność można mieć kiedy wartości nie zgadzają się z sensem fizycznym badanej zmiennej, dobrym przykładem jest ujemna wartość wieku. Ponadto w danych mogą występować braki, które należy odpowienio obsłużyć bądź przez usunięcie obserwacji zawierających brakujące wartości, bądź stosując jedną z metod uzupełniania braków.

W badanym zbiorze danych HOMA występuje 10 zmiennych: 
\begin{itemize}
\item Płeć, 
\item WIEK, 
\item BMI, 
\item WHR, 
\item TROJGLICERYDY, 
\item HOMA, 
\item LDL.HDL, 
\item FAT.ALL.P, 
\item FAT.A.P, 
\item FAT.G.P.
\end{itemize}


Zgodnie z opisem w treści zadania, w badaniu brało udział 194 kobiety i 84 mężczyzn w wieku od 20 do 40 lat. Sprawdzono, że wszystkie obserwacje znajdują się w tym przedziale z wyjątkiem jednej, dla której wiek wynosił -34 lata. Najprawdopodobniej wartość ta jest skutkiem omyłkowego pojawienia się znaku "minus", zatem obserwacji tej nie usunięto, a jedynie zmieniono wartość wieku na dodatnią.

Sprawdzono, że wszystkie wartości dotyczące poziomu tłuszczu w organizmie (całkowitego, gynoidalnego i androidalnego) zawierają się w sensownym przedziale [0, 100].

Zgodnie z treścią zadania, w badaniu brać powinny jedynie osoby o wskaźniku BMI zawierającym się w przedziale [18, 25]. Znaleziono w zbiorze 11 obserwacji które nie spełniały tego warunku. Usunięto je ze zbioru danych.

Sprawdzono, że wartości WHR, stężenia trójglicerydów i stosunku LDL/HDL wszystkie są nieujemne tak jak to być musi w rzeczywistości.

Po powyższych operacjach pozostało 265 obserwacji, z czego 259 kompletnych (bez braków). Liczba 6 wierszy z brakami stanowi $\frac{6}{259} \approx 2.3\%$ całego zbioru danych. Przy takiej ilości uzasadnione jest po prostu odrzucenie wierszy z brakującymi danymi zamiast podejmowania prób ich uzupełniania. W związku z powyższym usunięto 6 wierszy z brakami ze zbioru danych.

\subsection{Wstępna analiza danych}

\subsection{Wylosowanie danych testowych}
Ustawiono wartość ziarna i wylosowano 10 obserwacji, które nie uczestniczyły w dopasowywaniu modelu. Z danych wykorzystywanych w dalszym toku prac usunięto te 10 obserwacji otrzymując zbiór danych o liczności 249.

\subsection{Zbudowanie modelu regresji}
W oparciu o wszystkie numeryczne zmienne objaśniające zbudowano model regresji. Zmienną objaśnianą jest HOMA, zmiennymi objaśniającymi są: 
\begin{enumerate}
\item WIEK, 
\item BMI, 
\item WHR, 
\item TROJGLICERYDY, 
\item LDL.HDL, 
\item FAT.ALL.P, 
\item FAT.A.P, 
\item FAT.G.P.
\end{enumerate}
Czyli wszystkie z wyjątkiem płci, która jest zmienną kategoryczną a nie numeryczną.

\subsection{Analiza reszt dla pełnego modelu}

\subsection{Identyfikacja i usunięcie obserwacji odstających i wpływowych}

\subsection{Wybór modelu}

\subsection{Interpretacja współczynników wybranego modelu}

\subsection{ponowna analiza reszt dla wybranego modelu}


\section{Wnioski}

\end{document}







%\subsection{Wypunktowania}
%To jest wypunktowanie:
%\begin{itemize}
%\item punkt
%\item punkt
%\item punkt
%\end{itemize}
%\subsection{Wyliczenie}
%To jest wyliczenie:
%\begin{enumerate}
%\item pierwszy,
%\item drugi,
%\item trzeci.
%\end{enumerate}
%\subsection{Wykres}
%Na Rysunku~\ref{fig:wykres} znajduje siê przyk³adowy wykres.
%\begin{figure}[H]
%\includegraphics[width=\textwidth]{wykres}
%\caption{Przyk³adowy wykres}
%\label{fig:wykres}
%\end{figure}
%\subsection{Wzory}
%\begin{equation}
%2 + 2 = 4
%\end{equation}
%\begin{equation}
%E = mc^2
%\end{equation}
%\begin{equation}
%\left[- \frac{\hbar^2}{2M} + V(X) \right] \Psi(x)=  \mathcal{E} \Psi(x)
%\end{equation}